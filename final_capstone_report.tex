\documentclass[]{article}
\usepackage{lmodern}
\usepackage{amssymb,amsmath}
\usepackage{ifxetex,ifluatex}
\usepackage{fixltx2e} % provides \textsubscript
\ifnum 0\ifxetex 1\fi\ifluatex 1\fi=0 % if pdftex
  \usepackage[T1]{fontenc}
  \usepackage[utf8]{inputenc}
\else % if luatex or xelatex
  \ifxetex
    \usepackage{mathspec}
  \else
    \usepackage{fontspec}
  \fi
  \defaultfontfeatures{Ligatures=TeX,Scale=MatchLowercase}
\fi
% use upquote if available, for straight quotes in verbatim environments
\IfFileExists{upquote.sty}{\usepackage{upquote}}{}
% use microtype if available
\IfFileExists{microtype.sty}{%
\usepackage{microtype}
\UseMicrotypeSet[protrusion]{basicmath} % disable protrusion for tt fonts
}{}
\usepackage[margin=1in]{geometry}
\usepackage{hyperref}
\hypersetup{unicode=true,
            pdftitle={R Notebook},
            pdfborder={0 0 0},
            breaklinks=true}
\urlstyle{same}  % don't use monospace font for urls
\usepackage{color}
\usepackage{fancyvrb}
\newcommand{\VerbBar}{|}
\newcommand{\VERB}{\Verb[commandchars=\\\{\}]}
\DefineVerbatimEnvironment{Highlighting}{Verbatim}{commandchars=\\\{\}}
% Add ',fontsize=\small' for more characters per line
\usepackage{framed}
\definecolor{shadecolor}{RGB}{248,248,248}
\newenvironment{Shaded}{\begin{snugshade}}{\end{snugshade}}
\newcommand{\AlertTok}[1]{\textcolor[rgb]{0.94,0.16,0.16}{#1}}
\newcommand{\AnnotationTok}[1]{\textcolor[rgb]{0.56,0.35,0.01}{\textbf{\textit{#1}}}}
\newcommand{\AttributeTok}[1]{\textcolor[rgb]{0.77,0.63,0.00}{#1}}
\newcommand{\BaseNTok}[1]{\textcolor[rgb]{0.00,0.00,0.81}{#1}}
\newcommand{\BuiltInTok}[1]{#1}
\newcommand{\CharTok}[1]{\textcolor[rgb]{0.31,0.60,0.02}{#1}}
\newcommand{\CommentTok}[1]{\textcolor[rgb]{0.56,0.35,0.01}{\textit{#1}}}
\newcommand{\CommentVarTok}[1]{\textcolor[rgb]{0.56,0.35,0.01}{\textbf{\textit{#1}}}}
\newcommand{\ConstantTok}[1]{\textcolor[rgb]{0.00,0.00,0.00}{#1}}
\newcommand{\ControlFlowTok}[1]{\textcolor[rgb]{0.13,0.29,0.53}{\textbf{#1}}}
\newcommand{\DataTypeTok}[1]{\textcolor[rgb]{0.13,0.29,0.53}{#1}}
\newcommand{\DecValTok}[1]{\textcolor[rgb]{0.00,0.00,0.81}{#1}}
\newcommand{\DocumentationTok}[1]{\textcolor[rgb]{0.56,0.35,0.01}{\textbf{\textit{#1}}}}
\newcommand{\ErrorTok}[1]{\textcolor[rgb]{0.64,0.00,0.00}{\textbf{#1}}}
\newcommand{\ExtensionTok}[1]{#1}
\newcommand{\FloatTok}[1]{\textcolor[rgb]{0.00,0.00,0.81}{#1}}
\newcommand{\FunctionTok}[1]{\textcolor[rgb]{0.00,0.00,0.00}{#1}}
\newcommand{\ImportTok}[1]{#1}
\newcommand{\InformationTok}[1]{\textcolor[rgb]{0.56,0.35,0.01}{\textbf{\textit{#1}}}}
\newcommand{\KeywordTok}[1]{\textcolor[rgb]{0.13,0.29,0.53}{\textbf{#1}}}
\newcommand{\NormalTok}[1]{#1}
\newcommand{\OperatorTok}[1]{\textcolor[rgb]{0.81,0.36,0.00}{\textbf{#1}}}
\newcommand{\OtherTok}[1]{\textcolor[rgb]{0.56,0.35,0.01}{#1}}
\newcommand{\PreprocessorTok}[1]{\textcolor[rgb]{0.56,0.35,0.01}{\textit{#1}}}
\newcommand{\RegionMarkerTok}[1]{#1}
\newcommand{\SpecialCharTok}[1]{\textcolor[rgb]{0.00,0.00,0.00}{#1}}
\newcommand{\SpecialStringTok}[1]{\textcolor[rgb]{0.31,0.60,0.02}{#1}}
\newcommand{\StringTok}[1]{\textcolor[rgb]{0.31,0.60,0.02}{#1}}
\newcommand{\VariableTok}[1]{\textcolor[rgb]{0.00,0.00,0.00}{#1}}
\newcommand{\VerbatimStringTok}[1]{\textcolor[rgb]{0.31,0.60,0.02}{#1}}
\newcommand{\WarningTok}[1]{\textcolor[rgb]{0.56,0.35,0.01}{\textbf{\textit{#1}}}}
\usepackage{graphicx,grffile}
\makeatletter
\def\maxwidth{\ifdim\Gin@nat@width>\linewidth\linewidth\else\Gin@nat@width\fi}
\def\maxheight{\ifdim\Gin@nat@height>\textheight\textheight\else\Gin@nat@height\fi}
\makeatother
% Scale images if necessary, so that they will not overflow the page
% margins by default, and it is still possible to overwrite the defaults
% using explicit options in \includegraphics[width, height, ...]{}
\setkeys{Gin}{width=\maxwidth,height=\maxheight,keepaspectratio}
\IfFileExists{parskip.sty}{%
\usepackage{parskip}
}{% else
\setlength{\parindent}{0pt}
\setlength{\parskip}{6pt plus 2pt minus 1pt}
}
\setlength{\emergencystretch}{3em}  % prevent overfull lines
\providecommand{\tightlist}{%
  \setlength{\itemsep}{0pt}\setlength{\parskip}{0pt}}
\setcounter{secnumdepth}{0}
% Redefines (sub)paragraphs to behave more like sections
\ifx\paragraph\undefined\else
\let\oldparagraph\paragraph
\renewcommand{\paragraph}[1]{\oldparagraph{#1}\mbox{}}
\fi
\ifx\subparagraph\undefined\else
\let\oldsubparagraph\subparagraph
\renewcommand{\subparagraph}[1]{\oldsubparagraph{#1}\mbox{}}
\fi

%%% Use protect on footnotes to avoid problems with footnotes in titles
\let\rmarkdownfootnote\footnote%
\def\footnote{\protect\rmarkdownfootnote}

%%% Change title format to be more compact
\usepackage{titling}

% Create subtitle command for use in maketitle
\newcommand{\subtitle}[1]{
  \posttitle{
    \begin{center}\large#1\end{center}
    }
}

\setlength{\droptitle}{-2em}
  \title{R Notebook}
  \pretitle{\vspace{\droptitle}\centering\huge}
  \posttitle{\par}
  \author{}
  \preauthor{}\postauthor{}
  \date{}
  \predate{}\postdate{}


\begin{document}
\maketitle

Collecting the Data from Twitter: I'm going to describe the methods and
packages used in gathering the data for this study.

I leveraged the \emph{BotorNot} API to cull some \emph{bot} accounts
centered around the \emph{hashtag} \#Charlottesville. The API has a
Python interface and returns an XML frame.

library(reticulate) \#\# library(XML)

\hypertarget{path_to_python---usrlocalbinpython3.6}{%
\subsection{\#path\_to\_python \textless{}-
``/usr/local/bin/python3.6''}\label{path_to_python---usrlocalbinpython3.6}}

\hypertarget{path_to_python--}{%
\subsection{path\_to\_python \textless{}-}\label{path_to_python--}}

\begin{verbatim}
  "/Library/Frameworks/Python.framework/Versions/3.6/bin/python3.6"
\end{verbatim}

\hypertarget{use_pythonpath_to_python}{%
\subsection{use\_python(path\_to\_python)}\label{use_pythonpath_to_python}}

\hypertarget{dbotimportdebot}{%
\subsection{dbot=import(``debot'')}\label{dbotimportdebot}}

\hypertarget{db-dbotdebotyocr3mkac7u6hjkzrnnock2jjnwrqgsfwwyo8yob-this-is-my-api-key-cbotshtmlparsedbget_related_botscharlottesville}{%
\subsection{\texorpdfstring{db =
dbot\(DeBot('YocR3mKAc7U6hjKZrNnOCk2jjnWrqgSfwWYo8yOb') #This is my API key ## cbots=htmlParse(db\)get\_related\_bots(`Charlottesville'))}{db = dbotDeBot('YocR3mKAc7U6hjKZrNnOCk2jjnWrqgSfwWYo8yOb') \#This is my API key \#\# cbots=htmlParse(dbget\_related\_bots(`Charlottesville'))}}\label{db-dbotdebotyocr3mkac7u6hjkzrnnock2jjnwrqgsfwwyo8yob-this-is-my-api-key-cbotshtmlparsedbget_related_botscharlottesville}}

I then parsed the XML data into a dataframe:

\hypertarget{botlist---xmltolistcbots}{%
\subsection{botlist \textless{}-
xmlToList(cbots)}\label{botlist---xmltolistcbots}}

\hypertarget{librarylistviewer}{%
\subsection{library(listviewer)}\label{librarylistviewer}}

\hypertarget{flatbot---flattenbotlist}{%
\subsection{flatbot \textless{}-
flatten(botlist)}\label{flatbot---flattenbotlist}}

\hypertarget{flatterbot---flattenflatbot}{%
\subsection{flatterbot \textless{}-
flatten(flatbot)}\label{flatterbot---flattenflatbot}}

\hypertarget{removeflatbot}{%
\subsection{remove(flatbot)}\label{removeflatbot}}

\hypertarget{flatbot---flattenflatterbot}{%
\subsection{flatbot \textless{}-
flatten(flatterbot)}\label{flatbot---flattenflatterbot}}

\hypertarget{newbotlist---unnamesapplyflatbot-2}{%
\subsection{\texorpdfstring{newBotlist \textless{}-
unname(sapply(flatbot, \texttt{{[}},
2))}{newBotlist \textless{}- unname(sapply(flatbot, {[}, 2))}}\label{newbotlist---unnamesapplyflatbot-2}}

Then using the \emph{twitteR} package, I got the userdata associated
with the accounts.

\hypertarget{librarytwittr}{%
\subsection{library(twittR)}\label{librarytwittr}}

\hypertarget{botaccts---lookupusersnewbotlist}{%
\subsection{botAccts \textless{}-
lookupUsers(newBotlist)}\label{botaccts---lookupusersnewbotlist}}

\hypertarget{botacctlist---twlisttodfbotaccts}{%
\subsection{botAcctlist \textless{}-
twListToDF(botAccts)}\label{botacctlist---twlisttodfbotaccts}}

Those variables are:

\begin{Shaded}
\begin{Highlighting}[]
\NormalTok{botAcctlist <-}\StringTok{ }\KeywordTok{readRDS}\NormalTok{(}\StringTok{"capstone dataset/botAcctlist"}\NormalTok{)}


\KeywordTok{names}\NormalTok{(botAcctlist) }
\end{Highlighting}
\end{Shaded}

\begin{verbatim}
##  [1] "description"       "statusesCount"     "followersCount"   
##  [4] "favoritesCount"    "friendsCount"      "url"              
##  [7] "name"              "created"           "protected"        
## [10] "verified"          "screenName"        "location"         
## [13] "lang"              "id"                "listedCount"      
## [16] "followRequestSent" "profileImageUrl"
\end{verbatim}

We can see that several of the variables for the goal dataset are
already in place: statusesCount, followersCount, etc. Not all of the
final variables are in place as they will be determined by a
transformation of available data. In order to ensure the character set
is usable, I'll set it explicitly. For convenience, let's list the
available data structures here:

\begin{Shaded}
\begin{Highlighting}[]
\NormalTok{botAcctlist}\OperatorTok{$}\NormalTok{description <-}\StringTok{ }\KeywordTok{iconv}\NormalTok{(botAcctlist}\OperatorTok{$}\NormalTok{description, }\StringTok{"latin1"}\NormalTok{, }\StringTok{"ASCII"}\NormalTok{)}
\NormalTok{botAcctlist}\OperatorTok{$}\NormalTok{name <-}\StringTok{ }\KeywordTok{iconv}\NormalTok{(botAcctlist}\OperatorTok{$}\NormalTok{name, }\StringTok{"latin1"}\NormalTok{, }\StringTok{"ASCII"}\NormalTok{)}

\KeywordTok{str}\NormalTok{(botAcctlist)}
\end{Highlighting}
\end{Shaded}

\begin{verbatim}
## 'data.frame':    375 obs. of  17 variables:
##  $ description      : chr  "in order to be happy be contented of what God give you." "Joaquin's Wife Forever na kame\n@JaoquinRedReyes Follow me (04-02-16)\n@JoaquinRedReyes My 51st Follower\n\nI LOVE GOD" "" "young dumb broke highschool kid" ...
##  $ statusesCount    : num  2931 2613 4495 41202 7228 ...
##  $ followersCount   : num  126 217 15 1083 145 ...
##  $ favoritesCount   : num  6408 2384 454 14232 1301 ...
##  $ friendsCount     : num  578 396 114 957 103 885 443 358 146 465 ...
##  $ url              : chr  NA NA NA NA ...
##  $ name             : chr  "Margie" "BabyGirl" "Rowena Talania" NA ...
##  $ created          : POSIXct, format: "2013-10-20 02:13:13" "2016-01-23 11:10:22" ...
##  $ protected        : logi  FALSE FALSE FALSE FALSE FALSE FALSE ...
##  $ verified         : logi  FALSE FALSE FALSE FALSE FALSE FALSE ...
##  $ screenName       : chr  "15_margiecastro" "2001shaira" "78fb1da7564c40c" "AaaRDieeee" ...
##  $ location         : chr  "certified #KN# #2NE1 #FANS" "" "" "" ...
##  $ lang             : chr  "en" "en" "en" "en" ...
##  $ id               : chr  "1973517494" "4802624365" "3653941454" "2207227844" ...
##  $ listedCount      : num  1 0 2 5 0 10 0 13 6 3 ...
##  $ followRequestSent: logi  FALSE FALSE FALSE FALSE FALSE FALSE ...
##  $ profileImageUrl  : chr  "http://pbs.twimg.com/profile_images/876396144095485952/_o_wX4uK_normal.jpg" "http://pbs.twimg.com/profile_images/737845398153240576/XsQH_Cc9_normal.jpg" "http://pbs.twimg.com/profile_images/835394002291576832/OUGCAyZQ_normal.jpg" "http://pbs.twimg.com/profile_images/891305317891317760/1_LU3ksF_normal.jpg" ...
\end{verbatim}

\emph{purr} gives me the mapping function to cull recent tweets for
these accounts to determine lingusitic diversity later.

\hypertarget{librarystringr}{%
\subsection{library(stringr)}\label{librarystringr}}

\hypertarget{librarypurr}{%
\subsection{library(purr)}\label{librarypurr}}

\hypertarget{acctnames---row.names.data.framebotacctlist}{%
\subsection{acctNames \textless{}-
row.names.data.frame(botAcctlist)}\label{acctnames---row.names.data.framebotacctlist}}

\hypertarget{unlistedbots---unlistbotacctlistscreenname}{%
\subsection{unlistedBots \textless{}-
unlist(botAcctlist{[}{]}\$screenName)}\label{unlistedbots---unlistbotacctlistscreenname}}

\hypertarget{recenttweetsds---mapunlistedbots-searchtwitter.x-resulttype-recent}{%
\subsection{recentTweetsDS \textless{}- map(unlistedBots{[}{]},
\textasciitilde{}searchTwitteR(.x, resultType =
``recent'',}\label{recenttweetsds---mapunlistedbots-searchtwitter.x-resulttype-recent}}

\begin{verbatim}
    n=200))
\end{verbatim}

\hypertarget{recenttweetsdf---twlisttodfflattenrecenttweetsds}{%
\subsection{recentTweetsDF \textless{}-
twListToDF(flatten(recentTweetsDS))}\label{recenttweetsdf---twlisttodfflattenrecenttweetsds}}

To regularise the character set in the tweets:

\hypertarget{recenttweetsdftext---iconvrecenttweetsdftext-latin1-ascii}{%
\subsection{\texorpdfstring{recentTweetsDF\(text <- iconv(recentTweetsDF\)text,
``latin1'',
``ASCII'')}{recentTweetsDFtext \textless{}- iconv(recentTweetsDFtext, ``latin1'', ``ASCII'')}}\label{recenttweetsdftext---iconvrecenttweetsdftext-latin1-ascii}}

The metadata in the recent tweets dataframe are named as follows, and
the data structures are listed below:

\hypertarget{namesrecenttweetsdf}{%
\subsection{names(recentTweetsDF)}\label{namesrecenttweetsdf}}

{[}1{]} ``text'' ``favorited'' ``favoriteCount'' ``replyToSN''
``created''\\
{[}6{]} ``truncated'' ``replyToSID'' ``id'' ``replyToUID''
``statusSource'' {[}11{]} ``screenName'' ``retweetCount'' ``isRetweet''
``retweeted'' ``longitude''\\
{[}16{]} ``latitude''

\hypertarget{strrecenttweetsdf}{%
\subsection{str(recentTweetsDF)}\label{strrecenttweetsdf}}

`data.frame': 15764 obs. of 16 variables: \$ text : chr NA NA ``RT
@Espanto2001: Ate Kathryn Bernardo mentioned me in her live\ldots{}I
have accomplished all of my goals in life\ldots{}''\textbar{}
\textbf{truncated} NA \ldots{} \$ favorited : logi FALSE FALSE FALSE
FALSE FALSE FALSE \ldots{} \$ favoriteCount: num 0 0 0 0 0 0 0 0 0 0
\ldots{} \$ replyToSN : chr ``15\_margiecastro'' NA NA NA \ldots{} \$
created : POSIXct, format: ``2017-10-30 12:39:45'' ``2017-10-29
23:24:21'' \ldots{} \$ truncated : logi FALSE FALSE FALSE FALSE FALSE
FALSE \ldots{} \$ replyToSID : chr ``917535662080802816'' NA NA NA
\ldots{} \$ id : chr ``924979133611876353'' ``924778966744838146''
``924653161784078337'' ``923396568392200192'' \ldots{} \$ replyToUID :
chr ``1973517494'' NA NA NA \ldots{} \$ statusSource : chr ``Twitter for
Android'' ``Twitter for Android'' \ldots{} \$ screenName : chr
``katrinacathe'' ``15\_margiecastro'' ``15\_margiecastro''
``15\_margiecastro'' \ldots{} \$ retweetCount : num 0 441 3164 44 350
\ldots{} \$ isRetweet : logi FALSE TRUE TRUE TRUE TRUE FALSE \ldots{} \$
retweeted : logi FALSE FALSE FALSE FALSE FALSE FALSE \ldots{} \$
longitude : chr NA NA NA NA \ldots{} \$ latitude : chr NA NA NA NA
\ldots{}

Note that this introduces \emph{NA} entries into the dataframe. An
important note for later. To reiterate, this dataframe contains accounts
that have been classified as bots.

I want to deal with direct tweets rather than retweets, so these are
seperated. Further transformation is ensuring that the original tweets
are grouped by account.

\begin{Shaded}
\begin{Highlighting}[]
\KeywordTok{library}\NormalTok{(tidyverse)}
\KeywordTok{library}\NormalTok{(qdapRegex)}

\NormalTok{recentTweetsDF <-}\StringTok{ }\KeywordTok{readRDS}\NormalTok{(}\StringTok{"capstone dataset/recentTweetsDF"}\NormalTok{)}
\NormalTok{recentTweetsDF}\OperatorTok{$}\NormalTok{text <-}\StringTok{ }\KeywordTok{iconv}\NormalTok{(recentTweetsDF}\OperatorTok{$}\NormalTok{text, }\StringTok{"latin1"}\NormalTok{, }\StringTok{"ASCII"}\NormalTok{)}

\NormalTok{AppURL <-}\StringTok{ }\KeywordTok{rm_between_multiple}\NormalTok{(recentTweetsDF}\OperatorTok{$}\NormalTok{statusSource,}\StringTok{"<"}\NormalTok{,}\StringTok{">"}\NormalTok{)}

\NormalTok{recentTweetsDF}\OperatorTok{$}\NormalTok{App <-}\StringTok{ }\KeywordTok{unlist}\NormalTok{(AppURL)}

\NormalTok{appcount <-}\StringTok{ }\NormalTok{recentTweetsDF }\OperatorTok\StringTok{      }
\StringTok{    }\KeywordTok{group_by}\NormalTok{(App) }\OperatorTok\StringTok{ }
\StringTok{    }\KeywordTok{summarize}\NormalTok{(}\DataTypeTok{Count=}\KeywordTok{n}\NormalTok{()) }\OperatorTok\StringTok{ }
\StringTok{    }\KeywordTok{arrange}\NormalTok{(}\KeywordTok{desc}\NormalTok{(Count))}
\NormalTok{recentTweetsDF <-}\StringTok{ }\KeywordTok{left_join}\NormalTok{(recentTweetsDF,appcount)}

\NormalTok{recentTweetsDF}\OperatorTok{$}\NormalTok{BoN <-}\StringTok{ }\KeywordTok{ifelse}\NormalTok{(recentTweetsDF}\OperatorTok{$}\NormalTok{Count}\OperatorTok{>}\DecValTok{4}\NormalTok{,}\DecValTok{0}\NormalTok{,}\DecValTok{1}\NormalTok{)}

\NormalTok{shredDF <-}\StringTok{ }\NormalTok{recentTweetsDF }\OperatorTok\StringTok{ }\KeywordTok{select}\NormalTok{(screenName, App, Count, BoN)}
\end{Highlighting}
\end{Shaded}

\begin{Shaded}
\begin{Highlighting}[]
\NormalTok{botDS <-}\StringTok{ }\KeywordTok{data.frame}\NormalTok{(botAcctlist}\OperatorTok{$}\NormalTok{screenName, botAcctlist}\OperatorTok{$}\NormalTok{id, botAcctlist}\OperatorTok{$}\NormalTok{created, botAcctlist}\OperatorTok{$}\NormalTok{statusesCount, }\DataTypeTok{langDiv =} \DecValTok{0}\NormalTok{, }\DataTypeTok{mean_time_betwn_tweets =} \DecValTok{0}\NormalTok{, }\DataTypeTok{bot =} \DecValTok{0}\NormalTok{)}
\NormalTok{columnnames <-}\StringTok{ }\KeywordTok{names}\NormalTok{(botDS)}
\NormalTok{columnnames[}\DecValTok{1}\NormalTok{] <-}\StringTok{ "screenName"}
\NormalTok{columnnames[}\DecValTok{2}\NormalTok{] <-}\StringTok{ "ID"}
\NormalTok{columnnames[}\DecValTok{3}\NormalTok{] <-}\StringTok{ "acct_created"}
\NormalTok{columnnames[}\DecValTok{4}\NormalTok{] <-}\StringTok{ "statusesCount"}
\KeywordTok{colnames}\NormalTok{(botDS) <-}\StringTok{ }\NormalTok{columnnames}

\NormalTok{tweetsRT <-}\StringTok{ }\KeywordTok{filter}\NormalTok{(recentTweetsDF, isRetweet }\OperatorTok{==}\StringTok{ "TRUE"}\NormalTok{)}
\NormalTok{tweetsOPEN <-}\StringTok{ }\KeywordTok{filter}\NormalTok{(recentTweetsDF, isRetweet }\OperatorTok{!=}\StringTok{ "TRUE"}\NormalTok{)}
\CommentTok{#tweeters <- unique(tweetsOPEN$screenName)}
\CommentTok{#length(tweeters)}

\NormalTok{botDS <-}\StringTok{ }\KeywordTok{left_join}\NormalTok{(botDS, shredDF, }\DataTypeTok{by =} \StringTok{"screenName"}\NormalTok{)}

\NormalTok{botDS <-}\StringTok{ }\NormalTok{botDS[, }\KeywordTok{c}\NormalTok{(}\DecValTok{1}\NormalTok{, }\DecValTok{6}\NormalTok{, }\DecValTok{14}\NormalTok{, }\DecValTok{16}\NormalTok{, }\DecValTok{15}\NormalTok{, }\DecValTok{18}\NormalTok{, }\DecValTok{3}\NormalTok{, }\DecValTok{12}\NormalTok{, }\DecValTok{13}\NormalTok{, }\DecValTok{4}\NormalTok{, }\DecValTok{7}\NormalTok{, }\DecValTok{9}\NormalTok{, }\DecValTok{5}\NormalTok{, }\DecValTok{8}\NormalTok{, }\DecValTok{10}\NormalTok{, }\DecValTok{11}\NormalTok{)]}


\NormalTok{groupedTweetsDF <-}\StringTok{ }\NormalTok{tweetsOPEN }\OperatorTok
\StringTok{  }\KeywordTok{group_by}\NormalTok{(screenName) }\OperatorTok
\StringTok{  }\KeywordTok{filter}\NormalTok{(screenName }\OperatorTok\StringTok{ }\NormalTok{botDS}\OperatorTok{$}\NormalTok{screenName)}
\NormalTok{castoffDF <-}\StringTok{ }\KeywordTok{setdiff}\NormalTok{(tweetsOPEN, groupedTweetsDF)}

\CommentTok{#binning the account creation column}
\KeywordTok{table}\NormalTok{(}\KeywordTok{cut}\NormalTok{(botDS}\OperatorTok{$}\NormalTok{acct_created, }\DataTypeTok{breaks =} \StringTok{"6 month"}\NormalTok{))}
\end{Highlighting}
\end{Shaded}

library(quanteda)

\#using quanteda pkg langDiv \textless{}-
textstat\_lexdiv(dfm(groupedTweetsDF\$text))

botDiv \textless{}- groupedTweetsDF \%\textgreater{}\%
group\_by(screenName) summarise(meanDiv = mean(langDiv))

botDS \textless{}- left\_join(botDS, botDiv)
botDS\(langDiv <- botDS\)meanDiv

botDS \textless{}- data.frame(botAcctlist\(screenName, botAcctlist\)id,
botAcctlist\(created,  botAcctlist\)statusesCount, langDiv = 0,
mean\_time\_betwn\_tweets = 0, bot = 0) columnnames \textless{}-
names(botDS) columnnames{[}1{]} \textless{}- ``screenName''
columnnames{[}2{]} \textless{}- ``ID'' columnnames{[}3{]} \textless{}-
``acct\_created'' columnnames{[}4{]} \textless{}- ``statusesCount''
colnames(botDS) \textless{}- columnnames

tweetsRT \textless{}- filter(recentTweetsDF, isRetweet == ``TRUE'')
tweetsOPEN \textless{}- filter(recentTweetsDF, isRetweet != ``TRUE'')

groupedTweetsDF \textless{}- tweetsOPEN \%\textgreater{}\%
group\_by(screenName) \%\textgreater{}\% filter(screenName \%in\%
botDS\$screenName) castoffDF \textless{}- setdiff(tweetsOPEN,
groupedTweetsDF)

```


\end{document}
